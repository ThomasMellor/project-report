\section{Introduction}

Condensation--the phenomenon whereby a macroscropic number of particles occupy the ground state a system--has been of theoretical and experimental interest for physicists for almost a century. Bose Einstein Condensation (BEC), perhaps the most familiar example, was predicted to occur for a system of non-interacting Bosons at a sufficiently low temperature. This was observed in ultrocold atoms in the 1990s, but there are many other examples of condensation, such as Bardeen Cooper Srhreifer superconductivity. The excitations can even be quasi particles, for example int he case of magnons--electronic spin waves.

More recently, another type of quasiparticle, the exciton-polariton, formed from the coupling of electron-hole (exciton) pairs and photons, has been observed to undergo condensation in microcavities. However, to maintain the condensate, the continual pumping of a laser is required, inhibiting the system from attaining equilibrium. Because it cannot, many questions still remain as to the exact nature of the condensate and how the understanding of condensation derived from equilibrium physics must be modified. 

The aim of this project is to address some of these issues by confirming behaviour of the condensate expected from a theoretical analysis. In particular, when describing the condensate as a macroscopic wavefunction $\rho(r,t)e^{i\theta(r,t)}$, it is found that the long range behaviour of the condensate can be described strictly by the evolution of the phase $\theta$ through the Karder-Parisi-Zhang (KPZ) equation. Originally used to describe height growth, in its current context the compactness of the $\theta$, i.e. that it is limited between $0$ and $2\pi$, allows for the existence of topoligal defects which are crucial in explaining the physics.

By employing techniques from the renormalistion group, it is found that depending on the parameters of the equation there are effectively two regime. The first is described by Berezinskii–Kosterlitz–Thouless (BKT) physics, which is that of the XY model. The second, the KPZ phase. 
During the porject, the evolution of the condensate phase was simulated through the KPZ equation, to confirm the that in the different regime the behaviour was as expected. 



 



